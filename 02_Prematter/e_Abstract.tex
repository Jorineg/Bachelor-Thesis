\iflanguage{english}{
	\chapter*{Abstract}
	}{\iflanguage{ngerman}{
		\chapter*{Zusammenfassung}
		}{}
	}
%------------------------------
% GERMAN
%------------------------------
\begin{otherlanguage}{ngerman}
\section*{Zusammenfassung}
Stencil-Berechnungen sind ein grundlegender Bestandteil einer Vielzahl von wissenschaftlichen Disziplinen, darunter Klimamodellierung, Fluiddynamik und Materialwissenschaft. Ihre geringe arithmetische Intensität macht ihre effiziente Implementierung auf traditioneller Hardware jedoch notorisch schwierig.

Der Cerebras \acf{wse} ist eine neuartige Hardware-Architektur, die fast eine Million Rechenkerne auf einem einzigen Chip vereint. Ihr Potenzial für hocheffiziente Stencil-Implementierungen wurde bereits im Kontext von 3D-Stencils demonstriert, typischerweise durch die Abbildung der z-Dimension auf den Speicher der einzelnen Kerne. Effiziente Implementierungen für 2D-Stencils stellen jedoch weiterhin eine offene Herausforderung dar.

Diese Arbeit präsentiert und evaluiert einen neuartigen Implementierungsansatz für zweidimensionale, sternförmige Stencils auf der Cerebras-Hardware. Durch die Aufteilung der Problemdomäne in Kacheln (Tiling) und deren Abbildung auf die einzelnen Rechenkerne reduziert dieser Ansatz den Kommunikationsaufwand im Verhältnis zur Rechenlast erheblich. Wir entwickeln und evaluieren zwei Kernel: einen durchsatzoptimierten, gekachelten Kernel für große Problemdomänen und eine spezialisierte Einzelzellen-Implementierung für Gitter, die vollständig auf die physikalischen Dimensionen des \ac{wse} passen.

Unsere Implementierungen erreichen eine bemerkenswerte Leistung von bis zu 165 Tera-Zellen/s auf einem simulierten \ac{wse}-3, erzielen für bestimmte Problemgrößen 41\% der Spitzengleitkommaleistung des \ac{wse}-3 und übertreffen eine optimierte Devito-Implementierung auf einer NVIDIA H100 \ac{gpu} um mehr als drei Größenordnungen.

\section*{Schlüsselwörter:} \itshape \germankeywords
\end{otherlanguage}
%------------------------------
% ENGLISH
%------------------------------
\begin{otherlanguage}{english}
\section*{Abstract}
Stencil computations are fundamental to various disciplines including climate modeling, fluid dynamics and material science.
However, their low arithmetic intensity makes their efficient implementation on traditional hardware notoriously difficult.

The Cerebras \acf{wse} is a novel hardware architecture, featuring almost one million cores on a single chip. Its potential for highly efficient stencil implementations has already been demonstrated in the context of 3D stencils, typically by mapping the z-dimension to the memory within each core. Yet, efficient implementations for 2D stencils remain an open challenge.

This work presents and evaluates a novel implementation approach for 2D star-shaped stencils on Cerebras hardware. By tiling the problem domain and mapping these tiles to the individual cores, this approach significantly reduces the communication overhead, relative to the computational workload. We develop and evaluate two kernels: a high-throughput, tiled kernel for large problem domains and a specialized single-cell implementation for grids that fit entirely within the \ac{wse}'s physical dimensions.

Our implementations achieve remarkable performance on a simulated \ac{wse}-3 of up to 165 Tera-cells/s, reach 41\% of \ac{wse}-3's peak floating-point performance for certain problem sizes, and outperform an optimized Devito implementation on an NVIDIA H100 \ac{gpu} by over three orders of magnitude. 
\section*{Keywords:} \itshape \englishkeywords
\end{otherlanguage}