\iflanguage{english}{
	\chapter*{Abstract}
	}{\iflanguage{ngerman}{
		\chapter*{Zusammenfassung}
		}{}
	}
%------------------------------
% GERMAN
%------------------------------
\begin{otherlanguage}{ngerman}
\section*{Zusammenfassung}
Effiziente Stencil-Berechnungen sind entscheidend für Bereiche wie Materialwissenschaft, Klimamodellierung oder Aerodynamik. Neue Hardware-Beschleuniger, wie die Cerebras Wafer Scale Engine (\ac{wse}), bieten ein neues Programmierparadigma, das effizientere Stencil-Berechnungen im Vergleich zu traditionellen Ansätzen mit \acp{cpu} oder \acp{gpu} ermöglicht.

Diese Arbeit untersucht das Design, die Implementierung und die Leistungsanalyse von 2D sternförmigen Stencils auf der Cerebras \ac{wse}-Architektur. Zwei verschiedene Implementierungen werden entwickelt: eine latenz-optimierte, direkt abgebildete Implementierung für Radius-1-Stencils und eine flexible, gekachelte Implementierung für variable-Radius-Stencils.

Die Hauptbeiträge umfassen eine detaillierte Leistungsanalyse beider Implementierungen, die Untersuchung des Parameterraums des gekachelten Algorithmus und den Vergleich der \ac{wse}-Implementierungen mit hochoptimierten Codes für moderne Mehrkern-\acp{cpu} und High-End-\acp{gpu}. Zusätzlich wird ein vereinfachtes Leistungsmodell für die implementierten Stencil-Operatoren vorgestellt, das wichtige architektonische Merkmale der \ac{wse} berücksichtigt.

Die Ergebnisse zeigen erhebliche Leistungsvorteile der \ac{wse}-Architektur für Stencil-Berechnungen mit Beschleunigungen von bis zu 800x im Vergleich zu \ac{gpu}-Implementierungen für bestimmte Konfigurationen, was das Potenzial spezialisierter Hardware-Beschleuniger für rechnerische Wissenschaftsanwendungen unterstreicht.

\section*{Schlüsselwörter:} \itshape \germankeywords
\end{otherlanguage}
%------------------------------
% ENGLISH
%------------------------------
\begin{otherlanguage}{english}
\section*{Abstract}
Efficient stencil computation is essential to fields as material science, climate modeling or aero dynamics. New hardware accelerators, like the Cerebras wafer scale engine (\ac{wse}), offer a new programming paradigm that allows more efficient stencil computation compared to traditional approaches using \acp{cpu} or \acp{gpu}.

This thesis investigates the design, implementation, and performance analysis of 2D star-shaped stencils on the Cerebras \ac{wse} architecture. Two distinct implementations are developed: a latency-optimized, directly-mapped implementation for radius-1 stencils and a flexible, tiled implementation for variable-radius stencils. 

The main contributions include a detailed performance analysis comparing both implementations, investigating the tiled algorithm's parameter space, and benchmarking \ac{wse} implementations against highly optimized code for modern multi-core \acp{cpu} and high-end \acp{gpu}. Additionally, a simplified performance model for the implemented stencil operators is presented, accounting for key architectural features of the \ac{wse}.

The results demonstrate significant performance advantages of the \ac{wse} architecture for stencil computations, with speedups of up to 800x compared to \ac{gpu} implementations for certain configurations, highlighting the potential of specialized hardware accelerators for computational science applications.

\section*{Keywords:} \itshape \englishkeywords
\end{otherlanguage}