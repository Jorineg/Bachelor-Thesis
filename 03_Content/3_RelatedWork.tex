\chapter{Related Work}
The inherently low arithmetic intensity of stencil codes made their optimization a long-standing research topic.
Most work for traditional \ac{hpc} architectures like \ac{cpu} and \ac{gpu} focuses on cache optimization by using techniques like tiling, loop transformations or temporal blocking, which increases the effective arithmetic intensity and shift the problem more towards being compute-bound.
Tiling improves the cache utilization by splitting the problem domain spatially into parts, so that cached values can be used for as many updates as possible.
Tiling can be implemented hierarchically for all cache levels.
Temporal blocking takes the concept of tiling a step further by performing multiple iterations on a single tile before moving to the next tile.
Bringing all of these techniques together for a specific application is inherently complex and requires careful tuning, taking into account the specific hardware cache sizes, the problem size and the stencil layout.
This raised the need for specialized stencil compilers.
One of these is Devito, a Python based \ac{dsl} for finite difference stencil codes, that generates code for different \ac{hpc} architectures, incorporating various optimization techniques \cite{lange2016devito}. 
Other systems include Tiramisu, ExaStencils and Pochoir which use a divide-and-conquer algorithm to make it oblivious to the cache size \cite{baghdadi2019tiramisu,lengauer2014exastencils,tang2011pochoir}.

However, emerging hardware accelerators like the \ac{wse} introduce a vastly different programming model, where \acp{pe} are independent hardware units with no synchronized control, distributed memory, and no cache hierarchy.
As a result, most existing approaches are not directly applicable, leaving the performance-limiting factors of stencil codes on the \ac{wse} underexplored and difficult to target with general-purpose stencil compilers.

\citeauthor{rocki2020fast} \cite{rocki2020fast} were the first to use the \ac{wse} architecture for stencil computations and showcased an efficient 7-point 3D stencil for finite differences.
They established the foundation for replacing memory accesses in stencil patterns with communication between \acp{pe} that each hold a portion of the grid. In their approach, the z-dimension is mapped to memory within individual \acp{pe}, while the x and y dimensions are mapped across the \acp{pe} of the \ac{wse}-2, resulting in each \ac{pe} storing a 1D column of the grid.
Building on this work, \citeauthor{jacquelin2022scalable} \cite{jacquelin2022scalable} implement and analyze a more complex 25-point 3D stencil on the \ac{wse}-2. Unlike the work by \citeauthor{rocki2020fast} \cite{rocki2020fast}, this stencil requires multi-hop communication, which involves complex dynamic routing configurations.
Notably, they demonstrate near-ideal weak scaling: as the number of \acp{pe} increases from \numproduct{200 x 200} to the full \ac{wse}-2 dimensions of \numproduct{755 x 994}, and the problem size is scaled proportionally, the execution time remains effectively constant, with only a 1.5\% increase.

There have also been approaches working on abstracting away as much complexity as possible and to automatically generate \ac{csl} from a higher level \ac{api} or stencil \ac{dsl} as presented by \citeauthor{woo2022disruptive} \cite{woo2022disruptive} and \citeauthor{sai2024automated} \cite{sai2024automated}. However, these approaches focus on 3D stencils and while they might be able to generate code for 2D stencils, they do not explore topics like tiling of the x and y dimensions. While this idea is less practical for most 3D stencil applications, it is much more promising for 2D stencils and therefore the focus of our work. We furthermore systematically analyze the performance implications of different tile sizes and radii, directly comparing low-order and high-order stencil patterns on the \ac{wse}.