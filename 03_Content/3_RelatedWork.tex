\chapter{Related Work}
Optimizing stencil codes has been a long-standing research topic.
For traditional \ac{hpc} architectures, like \ac{cpu} and \ac{gpu}, where this problem is memory bound (!!!!!!!!!!!!!!! ooops but not for me !!!! see amd epyc !!!!), most work focuses on cache optimization by using techniques like tiling, loop reordering or temporal blocking.
\textcite{lange2016devito} introduced Devito, a python based \ac{dsl} for finite difference stencil codes, that generates code for different \ac{hpc} architectures, incorporating these techniques.

However, new hardware accelerators like the \ac{wse} feature a vastly different programming model with \acp{pe} representing different hardware actors with a lacking synchronized control instance and a completly missing cache hierarchy. This makes most studied ideas not applicable and the \ac{wse} so far too difficult to target for general-purpose stencil \ac{dsl}s like Devito.

\textcite{rocki2020fast} were the first to use the \ac{wse} architecture for stencil computations and layed the foundation for how the memory access in stencil patterns can be replaced by communication between \ac{pe}s holding parts of the grid.

\textcite{jacquelin2022massively} implement and analyze a 25-point 3D stencil on the \ac{wse}-2.
They do this by mapping the z-dimension of the grid to the \ac{pe}s memory while mapping the x and y dimensions to the \ac{wse}-2's \ac{pe}s, so that each \ac{pe} holds a 1D column of the grid.
Noticably they show almost perfect weak scaling with the an increasing number of \ac{pe}s, meaning that when the number of \ac{pe}s is increased from \numproduct{200 x 200} to the full \ac{wse}-2's maximum dimension of \numproduct{755 x 994} while increasing the problem size by the same factor, the runtime is almost constant with only a 1.5\% increase.

There have also been apporaches for compiling higher level stencil languages to code for the \ac{wse} as presented by \textcite{woo2022disruptive} and \textcite{sai2024automated}. However, these approaches focus on 3D stencils and while they might be able to generate code for 2D stencils, they don't explore topics like tiling of the x and y dimensions. While this idea is less practical for most 3D stencil applications, it is a lot more promising for 2D stencils and therefore the focus of our work.

Existing work analyzes only one specific stencil pattern at a time and does not explore the effects of different radii, directly comparing the low-order and high-order stencil patterns on the \ac{wse}.