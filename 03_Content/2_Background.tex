\chapter{Background}
etwas mehr als man braucht um die arbeit zu verstehen. Aber nicht viel mehr.
To me before I wrote this:
\subsection{Stencil computation: what is it used for, what can it do}
The focus of this work is a 2d star-shaped stencil. There is an underlying 2d grid of values of size mxn.  
2d star shaped stencil with radius 1 is discretization of the Laplacian operator.
Used to solve:
\begin{itemize}
    \item Heat Equation: describes how temperature T diffuses through a material over time t. Used e.g. for Semiconductor Chip Design and Materials Science
    \item Poisson's Equation: relates the electric potential φ to a charge distribution ρ. It's used to find the voltage field in a region given a set of fixed charges. Used e.g. for Sensor Design and Particle Accelerators.
\end{itemize}
Different border critiera. Widely used is dicrilet border which we focus on in this work.
Jacobi method and Gauss-Seidel method. we implement Jacobi.
\subsection{Cerebras hardware}
The Cerebras \ac{WSE} has very distinct characteristics not only compared to \acp{CPU} but also to \acp{GPU}.
Instead of seperated compute cores and memory, Cerebras \ac{WSE} features several hundred thousand Compute elements that each consist of \qty 
Hardware: \ac{pe} = router + \ac{ce}
wse2 + wse3, grid like arangement of \acp{pe}, number of \acp{pe}, memory per \ac{pe}, links between \acp{pe}, colors, simd operations, data structure descriptiors, data structure registers, input and output queues, limitations (e.g. cannot receive and send at same time, limited number of colors, input output queues and dsrs)
memory banks, bank conflicts
tasks, and task activation for async communication, task blocking
clock speed
\subsection{Stencil algorithms for Cerebras}
some previous work for 3d stencils, cerebras example.
cerebras is ideal harware for stencil computation. 
there is much work on 3d stgencils. nothing yet on 2d 