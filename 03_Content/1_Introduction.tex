\chapter{Introduction}

linear 2d star-shaped stencils. 

Stencil computations form the computational heart of scientific simulations across disciplines, from modeling heat dissipation in semiconductor chips to simulating complex seismic waves.

- motivation: why are stencil coputations important 
- problem: why is efficient implementation on cerebres difficult, what are the chances?

challenges:
- Cerebras uses \ac{csl} for the \ac{wse} chip. no compiler for higher level languages like stencil dsls to \ac{csl}
- also: prgoramming paradigm is vastly differnet. no shared memory. no cache. each \ac{pe} completely indendent from others.

-> needs hand crafted kernel/prgramm/solution (?) for every problem

difficulties:
- hardware new and limited access
- language new, not widely adopted, very limited documentation

We identify following research questions:
\begin{itemize}
    \item How can 2D star-shaped stencils be mapped efficiently to the Cerebras \ac{wse}-Architecture and what are the key design choices?
    \item How does the performance of a specialized (radius 1, non-tiled) implementation scale compared to a generalized (tiled, variable radius) implementation on the \ac{wse}?
    \item How does the performance of the Cerepbras-Implentation position itself compared to highly optimized implementations on traditional \ac{hpc}-Arcitectures (\ac{cpu}, \ac{gpu})? 
\end{itemize}

Der job der Einleitung: die contributions verstehen.
(mehrere stencil paper introductions lesen)
muss nicht mehr als 2 oder 3 seiten sein.

The main contributions of this thesis are:
\begin{itemize}
    \item Two different implementations for linear two-dimensional star-shaped stencils on the Cerebras \ac{wse}, a single cell implementation optimized for latency, and a tiled implementation optimized for throughput and larger grid sizes (Chapter~\autoref{sec:implementation})
    \item A simplified performance model for the implemented stencil operators on the \ac{wse}, accounting for key architectural features (Chapter~\autoref{sec:theory_performance})
    \item A performance evaluation investigating the effects of different tile sizes and radii for the Cerebras \ac{wse}-2 and \ac{wse}-3 architectures, and a comparison to optimized code generated from the high-level stencil \ac{dsl} Devito targeting AMD Epyc \ac{cpu} and Nvidia H100 \ac{gpu} (Chapter~\autoref{sec:experiments})
\end{itemize}
