\chapter{Auto-completion}
Texmaker (and propably many other editors) offer the possibility to define additional commands for automatic completion. This means they are suggested when you type a command. They can be edited under User $\rightarrow$ Customize Completion. The commands, which are assumed to be used on a regular basis in a thesis, are stated in \autoref{tab:auto_commands}. You can simply copy each line and add it to your Texmaker. This takes approximately five minutes and saves you a lot of time when you actually write something, especially units.
\begin{longtable}{l p{0.45\textwidth}}
	\caption{List of recommended auto-complete commands in Texmaker}\label{tab:auto_commands}\\
\toprule
Command					& Explanation								\\
\midrule
\endfirsthead
	%
	\multicolumn{2}{c}%
	{ \itshape \tablename~\thetable\ (continued).}                  \\
	\midrule
Command					& Explanation								\\	
	\midrule
	\endhead
	%
	\midrule
	\multicolumn{2}{r}{{Continued on next page}}                    \\
	\endfoot
	%
	\bottomrule
	\endlastfoot
	% table content
	\verb+\ac{@}+						&	Acronym in text 						\\
	\verb+\ampere+						&	\unit{\ampere} unit 					\\
	\verb+\Autoref{#label#}+			&	Autoref with capitalized first letter	\\
	\verb+\bar+							&	\unit{\bar} unit						\\
	\verb+\begin{algorithm}+			&	new algorithm							\\
	\verb+\begin{definition}+			&	new definition							\\
	\verb+\begin{lemmaenv}+				&	new lemma								\\
	\verb+\begin{longtable}{@}+			&	new long table							\\
	\verb+\begin{mdframed}+				&	new frame								\\
	\verb+\begin{overpic}[@]{@}+		&	new frame								\\
	\verb+\begin{remarkenv}+			&	new remark								\\
	\verb+\begin{tablenotes}+			&	table notes in three part table			\\
	\verb+\begin{theoremenv}+			&	new theorem								\\
	\verb+\begin{threeparttable}+		&	new three part table					\\
	\verb+\bottomrule+					&	bottom rule in tabulars					\\
	\verb+\celsius+						&	\unit{\celsius} unit					\\
	\verb+\ch{@}+						&	new chemical formula					\\
	\verb+\cubic+						&	for cubed unit							\\
	\verb+\enquote{@}+					&	new quote in current language			\\
	\verb+\gram+						&	\unit{\gram} unit						\\
	\verb+\joule+						&	\unit{\joule} unit						\\
	\verb+\kelvin+						&	\unit{\kelvin} unit						\\
	\verb+\kilo+						&	for kilo in units						\\
	\verb+\mega+						&	for mega in units						\\
	\verb+\metre+						&	\unit{\metre} unit						\\
	\verb+\midrule+						&	mid rule in tabulars					\\
	\verb+\milli+						&	for milli in units						\\
	\verb+\missingfigure[@]{@}+			&	new missing figure with options			\\
	\verb+\missingfigure{@}+			&	new missing figure without options		\\
	\verb+\mole+						&	\unit{\mole} unit						\\
	\verb+\myfigure[@][@]{@}[@]{@}[@]+	&	new figure								\\
	\verb+\nomenclature{@}{@}{@}{@}+	&	new symbol								\\
	\verb+\num{@}+						&	new number								\\
	\verb+\parencite[@]{#bib#}+			&	new paren cite with options				\\
	\verb+\parencite{#bib#}+			&	new paren cite without options			\\
	\verb+\pascal+						&	\unit{\pascal} unit						\\
	\verb+\pder[@][@]{@}+				&	partial derivative						\\
	\verb+\per+							&	division command in units				\\
	\verb+\qty{@}{@}+					&	new number with unit					\\
	\verb+\qtyrange{@}{@}{@}+			&	new range for units						\\
	\verb+\roundbrack{@}+				&	round brackets around argument			\\
	\verb+\squared+						&	for squared unit						\\
	\verb+\textcite[@]{#bib#}+			&	new text cite with option				\\
	\verb+\textcite{#bib#}+				&	new text cite without option			\\
	\verb+\todo[@]{@}+					&	new todo with option					\\
	\verb+\todo{@}+						&	new todo without option					\\
	\verb+\toprule+						&	top rule in tabulars					\\
	\verb+\tothe{@}+					&	for power in units						\\
	\verb+\unit{@}+						&	new unit								\\
	\verb+\verb+						&	for verbat output						\\
	\verb+\volt+						&	\unit{\volt} unit						\\
	\verb+\watt+						&	\unit{\watt} unit						\\
\end{longtable}