%------------------------------
% MICROTYPE
%------------------------------
\microtypecontext{spacing=nonfrench}
%------------------------------
% DIRECTORIES
%------------------------------
\setcounter{secnumdepth}{4}
\setcounter{tocdepth}{4}
%------------------------------
% DATE
%------------------------------
\makeatletter					% new date type of month and year only
\newcommand{\monthyeardate}{%
	\iflanguage{english}{\DTMenglishmonthname{\@dtm@month} \@dtm@year}{%
  \DTMgermanmonthname{\@dtm@month} \@dtm@year
  }
}
\makeatother	
%------------------------------
% FIGURES, TABLES, LISTS, NUMERATIONS, COUNTER
%------------------------------
\graphicspath{{Figures/}} 		% directory in which LaTeX searches for figures
\newcounter{includepdfpage}		% counter for pdfpages page references
\newlist{myList}{itemize}{1}
\setlist[myList]{leftmargin=0.0cm,nosep,label={}}
\newcolumntype{s}{>{\collectcell\unit}c<{\endcollectcell}}	% column type for SI units
\renewcommand{\topfraction}{0.98} 
\renewcommand{\bottomfraction}{0.98} 
%------------------------------
% DEDICATION
%------------------------------
\newenvironment{dedication_env}	% defines the dedication environment
{\Large\begin{quote}\begin{center}\begin{em}}
{\par\end{em}\end{center}\end{quote}}
%------------------------------
% HEADER
%------------------------------
\lehead{\scshape\leftmark}		% left-even-head
\cehead{}						% center-even-head
\rehead{}						% right-even-head
\lohead{}						% left-odd-head
\cohead{}						% center-odd-head
\rohead{\scshape\rightmark}		% right-odd-head
%------------------------------
% FOOTER
%------------------------------
\lefoot[\pagemark]{\pagemark}	% left-even-foot
\cefoot{}						% center-even-foot
\refoot{}						% right-even-foot
\lofoot{} 						% left-odd-foot
\cofoot{}						% center-odd-foot
\rofoot[\pagemark]{\pagemark}	% right-odd-foot
\renewcommand*{\chapterpagestyle}{plain}
%------------------------------
% LISTS OF ...
%------------------------------
\makeatletter					%
\renewcommand*{\@pnumwidth}{2.5em}% column width for page numbers
\makeatother					%
%------------------------------
% LIST OF ABBREVIATIONS
%------------------------------
\newlength{\myabbrevlength}
\setlength{\myabbrevlength}{2cm}
\newlength{\mydescriptionwidth}
\setlength{\mydescriptionwidth}{\linewidth-2\tabcolsep-\myabbrevlength}
\iflanguage{english}{
	\newcommand{\abbrevname}{List of Abbreviations}
    }{\iflanguage{ngerman}{
    	\newcommand{\abbrevname}{Abkürzungsverzeichnis}}{}
}
\NewAcroTemplate[list]{dbtasupertabu}{
	\AcroNeedPackage{array}%
	\def\arraystretch{1.2}
    \acronymsmapF{%
		\AcroAddRow{%
			\leavevmode\acrowrite{short} & \acrowrite{list} \acropagefill \tabularnewline}
    }
	{\AcroRerun}%
	\acroheading
	\acropreamble
    \begin{supertabular}{>{\bfseries\sffamily}p{\myabbrevlength}p{\mydescriptionwidth}@{}}
		\AcronymTable
	\end{supertabular}
}
\acsetup{	
	format/first-long=\itshape, % first appearence is set in italics
	index/use=true,				% add abbrevs to index
	make-links=true				% make links for abbrevs
}
%------------------------------
% TODOS
%------------------------------
\makeatletter
\iflanguage{ngerman}{
	\renewcommand{\@todonotes@todolistname}{ToDo-Verzeichnis}
}{
\iflanguage{english}{
	\renewcommand{\@todonotes@todolistname}{List of ToDos}
}{}
}
\makeatother
%------------------------------
% THEOREM STYLE
%------------------------------
\newtheoremstyle{mytheoremstyle}	% own theorem style
{\topsep}   
{\topsep}   
{\itshape}  
{0pt}       
{\bfseries} 
{}         
{5pt plus 1pt minus 1pt} 
{\thmname{#1} \thmnumber{#2} \thmnote{\ (#3)}:}
%------------------------------
% COMMANDS FOR THEOREM, ...
%------------------------------
\newcommand{\theoremname}{}
\newcommand{\lemmaname}{}
\newcommand{\remarkname}{}
\iflanguage{english}{
	\renewcommand{\theoremname}{Theorem}
	\renewcommand{\lemmaname}{Lemma}
	\renewcommand{\remarkname}{Remark}
	}{\iflanguage{ngerman}{
		\renewcommand{\theoremname}{Satz}
		\renewcommand{\lemmaname}{Hilfssatz}
		\renewcommand{\remarkname}{Bemerkung}
	}{}
	}
%------------------------------
% SET THEOREM STYLE
%------------------------------
\theoremstyle{mytheoremstyle}	% definition via amsthm package
\newtheorem{theorem}{\theoremname}
\newtheorem{lemma}{\lemmaname}
\newtheorem{remark}{\remarkname}
%------------------------------
% THEOREM ENVIRONMENT
%------------------------------
\newmdtheoremenv[
  linecolor=black,
  roundcorner=2pt,
  linewidth=2pt
]{theoremenv}{\theoremname}[chapter]
%------------------------------
% LEMMA ENVIRONMENT
%------------------------------
\newmdtheoremenv[
  linecolor=blue,
  roundcorner=2pt,
  linewidth=1pt
]{lemmaenv}{\lemmaname}[chapter]
%------------------------------
% REMARK ENVIRONMENT
%------------------------------
\newmdtheoremenv[
  linecolor=red,
  roundcorner=2pt,
  linewidth=0.5pt
]{remarkenv}{\remarkname}[chapter]
%------------------------------
% DEFINITION ENVIRONMENT
%------------------------------
\newmdtheoremenv[
  linecolor=black,
  roundcorner=2pt,
  linewidth=1pt
]{definition}{Definition}[chapter]
%------------------------------
% MATH AND OPERATORS
%------------------------------
\DeclareMathOperator{\rot}{rot}	% declares the rot math operator
\newcommand{\m}{\text{ mod }}
\newcommand*\diff{\mathop{}\!\textrm{d}}	% differential d, e.g. in integrals
%------------------------------
% UNITS
%------------------------------
\newcommand{\SIRangePhrase}{}
\sisetup{
	detect-family=true,			% adapt unit typesetting to surrounding font family
	detect-mode=true,			% adapt output to surrounding text style
	mode=text,					% text mode 
	range-units=single,			% it is 4 to 10 K and not 4K to 10K
	retain-explicit-plus,		% \SI{1E+4}{} = 1E+04 in the text
	retain-unity-mantissa=false,	% no mantissa for numbers like 1e4
	retain-zero-exponent=false,	% no exponent for numbers like 1e0
	separate-uncertainty=true,	% uncertainties are separated from numbers
}
\DeclareSIUnit\torr{Torr}		% defines the unit Torr
\DeclareSIUnit\fahrenheit{\degree F}% defines the unit °F
\DeclareSIUnit\atm{atm}			% defines the unit atm
\DeclareSIUnit\anno{a}			% defines the unit a for year
\DeclareSIUnit\bar{bar}			% defines the unit bar
\DeclareSIUnit\kJmol{\kilo\joule\per\mole} % defines the unit kJ/mol
\DeclareSIUnit\kJmolK{\kilo\joule\per\mole\per\kelvin} % defines the unit kJ/(molK)
%------------------------------
% CHEMISTRY
%------------------------------
\chemsetup{%
	formula=chemformula,			% use the chemformula package for typesetting formulas
	greek=upgreek,				% upright typesetting according to upgreek package
}
%------------------------------
% COLORS
%------------------------------
\definecolor{dbta_blue}{RGB}{51,102,153}
\definecolor{tub_red}{RGB}{197,14,31}
%------------------------------
% XPARSE
%------------------------------
% O is an optional argument and {} is the default, m ist the must-have argument
\NewDocumentCommand{\pder}{ O{} O{} m }{\dfrac{\partial^{#2}#1}{\partial#3^{#2}}}				% a short command for partial derivatives
\NewDocumentCommand{\roundbrack}{ m }{\left(#1 \right)} % a short command for round brackets
\NewDocumentCommand{\squarebrack}{ m }{\left[#1 \right]}% a short command for squared brackets
\NewDocumentCommand{\ind}{m O{}}{\mathrm{#1}\mathnormal{,#2}}			% a short command for indices, the first argument is printed as text, the second in math mode
%------------------------------
% CODE
%------------------------------
\renewcommand{\lstlistingname}{Code}	% rename listings
%------------------------------
% ENUMERATE ENVIRONMENT
%------------------------------
\renewcommand{\labelenumi}{\arabic{enumi})} 				% 1)
\renewcommand{\labelenumii}{\arabic{enumi}.\arabic{enumii}}	% 1.1)
\renewcommand{\labelenumiii}{\arabic{enumi}.\arabic{enumii}.\arabic{enumiii}}	% 1.1.1)
\setlist[enumerate,1]{%
	before*=\onehalfspacing,		% onehalfspacing in environments
	itemsep=0.15\baselineskip,	% vertical space between items
	labelindent=\parindent,		% labels are indented by this length
}%
%------------------------------
% ITEMIZE ENVIRONMENT
%------------------------------
\setlist[itemize,1]{%
	label={--},					% standard label is --
	before*=\onehalfspacing,		% onehalfspacing in environments
	itemsep=0.15\baselineskip,	% vertical space between items
	labelindent=\parindent,		% labels are indented by this length
}%
%------------------------------
% GERMAN CAPTION REDEFINITIONS
%------------------------------
\addto\captionsngerman{%		
\renewcommand{\figurename}{Abb.}	% changes german figure caption to Abb. chapter.number
\renewcommand{\tablename}{Tab.} 	% changes german table caption to Tab. chapter.number
\SetAlgorithmName{Algorithmus}{Algorithmus}{Algorithmenverzeichnis}	% german name of list of algorithms
\renewcommand{\nomname}{Symbolverzeichnis}	% german name of list of symbols		
\renewcommand{\lstlistlistingname}{Codeverzeichnis} % german name of list of codes
\def\indexname{Indexverzeichnis}
}
%------------------------------
% ENGLISH CAPTION REDEFINITIONS
%------------------------------
\addto\captionsenglish{	
\renewcommand{\figurename}{Fig.}	% changes english figure caption to Fig. chapter.number
\renewcommand{\tablename}{Tab.}	% changes english table caption to Tab. chapter.number
\renewcommand{\nomname}{List of Symbols} % english name of list of symbols
\renewcommand{\lstlistlistingname}{List of Codes} % english name of list of codes
\def\indexname{Index}
}
%------------------------------
% ADD TOCLIST ALGORITHM
%------------------------------
\addtotoclist[algorithm2e]{loa}
\setuptoc{loa}{totoc}
\newcommand*{\listofloaname}{\listalgorithmcfname}
\renewcommand{\listofalgorithms}{\listoftoc{loa}}
%------------------------------
% AUTOREF GERMAN
%------------------------------
\addto\extrasngerman{%
	\def\chapterautorefname{Kap.}%
	\def\sectionautorefname{Abschn.}%
	\def\subsectionautorefname{Abschn.}%
	\def\subsubsectionautorefname{Abschn.}%
	\def\figureautorefname{Abb.}%
	\def\tableautorefname{Tab.}%
	\def\theoremenvautorefname{Satz}%
	\def\lemmaenvautorefname{Hilfssatz}%
	\def\remarkenvautorefname{Bem.}%
	\def\definitionautorefname{Def.}%
	\def\algorithmautorefname{Alg.}%
	\def\assumptionautorefname{Ann.}%
	\def\equationautorefname~#1\null{Gl.~(#1)\null}%
	\def\chapterautorefnamelong{Kapitel}%
	\def\sectionautorefnamelong{Abschnitt}%
	\def\subsectionautorefnamelong{Abschnitt}%
	\def\subsubsectionautorefnamelong{Abschnitt}%	
	\def\figureautorefnamelong{Abbildung}%
	\def\tableautorefnamelong{Tabelle}%
	\def\equationautorefnamelong~#1\null{Gleichung~(#1)\null}%
	\def\theoremenvautorefnamelong{Satz}%
	\def\lemmaenvautorefnamelong{Hilfssatz}%
	\def\remarkenvautorefnamelong{Bemerkung}%
	\def\definitionautorefnamelong{Definition}%
	\def\algorithmautorefnamelong{Algorithmus}%
	\renewcommand{\SIRangePhrase}{~bis~}%
	\sisetup{%
	locale = DE,%
	range-phrase={\SIRangePhrase},	% for unit range "4 bis 10 K"
	}%	
}
%------------------------------
% AUTOREF ENGLISH
%------------------------------
\addto\extrasenglish{%
	\def\chapterautorefname{Chap.}%
	\def\sectionautorefname{Sec.}%
	\def\subsectionautorefname{Sec.}%
	\def\subsubsectionautorefname{Sec.}%	
	\def\figureautorefname{Fig.}%
	\def\tableautorefname{Tab.}%
	\def\equationautorefname~#1\null{Eq.~(#1)\null}%
	\def\theoremenvautorefname{Theorem}%
	\def\lemmaenvautorefname{Lemma}%
	\def\remarkenvautorefname{Remark}%
	\def\definitionautorefname{Def.}%
	\def\algorithmautorefname{Alg.}%
	\def\assumptionautorefname{Ass.}%
	\def\chapterautorefnamelong{Chapter}%
	\def\sectionautorefnamelong{Section}%
	\def\subsectionautorefnamelong{Section}%
	\def\subsubsectionautorefnamelong{Section}%	
	\def\figureautorefnamelong{Figure}%
	\def\tableautorefnamelong{Table}%
	\def\equationautorefnamelong~#1\null{Equation~(#1)\null}%
	\def\theoremenvautorefnamelong{Theorem}%
	\def\lemmaenvautorefnamelong{Lemma}%
	\def\remarkenvautorefnamelong{Remark}%
	\def\definitionautorefnamelong{Definition}%
	\def\algorithmautorefnamelong{Algorithm}%
	\renewcommand{\SIRangePhrase}{~to~}%
	\sisetup{%
	locale = US,%
	range-phrase={\SIRangePhrase},	% for unit range "4 to 10 K"
	}%	
}
%------------------------------
% AUTOREF (LONG FORM)
%------------------------------
\newcommand{\Autoref}[1]{%
  \begingroup%
  \renewcommand{\chapterautorefname}{\chapterautorefnamelong}%
  \renewcommand{\sectionautorefname}{\subsectionautorefnamelong}%
  \renewcommand{\subsectionautorefname}{\subsectionautorefnamelong}%
  \renewcommand{\subsubsectionautorefname}{\subsubsectionautorefnamelong}%
  \renewcommand{\figureautorefname}{\figureautorefnamelong}%
  \renewcommand{\tableautorefname}{\tableautorefnamelong}%
  \renewcommand{\equationautorefname}{\equationautorefnamelong}%
  \renewcommand{\theoremenvautorefname}{\theoremenvautorefnamelong}%
  \renewcommand{\lemmaenvautorefname}{\lemmaenvautorefnamelong}%
  \renewcommand{\remarkenvautorefname}{\remarkenvautorefnamelong}%
  \renewcommand{\definitionautorefname}{\definitionautorefnamelong}%
  \renewcommand{\algorithmautorefname}{\algorithmautorefnamelong}%
  \renewcommand{\assumptionautorefname}{\assumptionautorefnamelong}%
  \autoref{#1}%
  \endgroup%
}
%------------------------------
% REFERENCES AND CITATIONS
%------------------------------
\renewcaptionname{english}{\bibname}{List of References}
\renewcaptionname{ngerman}{\bibname}{Literaturverzeichnis}
\setlength\bibitemsep{0.5\baselineskip}	% distance between two references
\setlength{\bibhang}{5ex}		% indent below first line of reference
\DeclareFieldFormat[				% no quotation marks in citetitle
	article,
	inbook,
	incollection,
	inproceedings,
	patent,
	thesis,
	unpublished
]{citetitle}{#1\midsentence\isdot}						
\DeclareFieldFormat[				% no quotation marks in title
	article,
	inbook,
	incollection,
	inproceedings,
	patent,
	thesis,
	unpublished
]{title}{#1\midsentence\isdot} 	
\DeclareFieldFormat[			% remove capitalization in title and use grammar of a sentence instead
article,
]{titlecase:title}{\MakeSentenceCase*{#1}}
\DeclareFieldFormat[			% no quotation marks in title
	inproceedings,
]{booktitle}{\itshape #1}
\renewcommand*{\nameyeardelim}{\addcomma\space}			% comma between author and year	
\DeclareFieldFormat{pages}{#1}	% no S. or pp. before pages
\DeclareFieldFormat{type}{\itshape #1}	% print bachelor or master thesis in italics
\DefineBibliographyStrings{english}{% for backreferencing in list of references
  backrefpage = {cit. on p.},	
  backrefpages = {cit. on pp.},
}
\DefineBibliographyStrings{ngerman}{%
  backrefpage = {zit. auf S.},	
  backrefpages = {zit. auf den S.},	
}
\renewbibmacro{in:}{%			% suppress in: before name of journal, book, etc.
  \ifboolexpr{%
     test {\ifentrytype{article}}%
     or
     test {\ifentrytype{book}}%
     or
     test {\ifentrytype{inproceedings}}%     
  }{}{\printtext{\bibstring{in}\intitlepunct}}%
}							
\DefineBibliographyStrings{ngerman}{% before URL german
urlseen = {letzter Zugriff},
}
\DefineBibliographyStrings{english}{% before URL english
urlseen = {last access},
}
\renewbibmacro*{volume+number+eid}{	% how volume and issue are printed in list of references
  \printfield{volume}
  \printfield[parens]{number}%
  \setunit{\addcomma\space}%
  \printfield{eid}
}
\DeclareDelimFormat[bib]{nametitledelim}{\addcolon\space} % use colon after year in list of references
\DeclareDelimFormat{finalnamedelim}{\addcomma\space}	  % delimiter between authors	
\DeclareDelimFormat[textcite]{finalnamedelim}{			  % delimiter for textcite
  \addspace\&\space}
\DeclareDelimFormat[parencite]{finalnamedelim}{			  % delimiter for parencite
  \addspace\bibstring{and}\space}

\renewbibmacro*{begentry}{%
	\iffieldundef{shorthand}{}{%
		\printfield{shorthand}\addspace ---\addspace
	}
}
\DeclareSortingTemplate{nyt}{%		sorting template for shorthand
	\sort{%
		\field{presort}%
	}%
	\sort[final]{%
		\field{sortkey}%
	}%
	\sort{%
		\field{shorthand}%
		\field{author}%
		\field{editor}%
		\field{title}%
	}%
	\sort{%
		\field{year}
	}%
	\sort{%
		\field{title}%
	}%
}

\renewbibmacro*{textcite}{%			redefines textcite so that shorthands also work with it
    \ifnameundef{labelname}{%
    	\iffieldundef{shorthand}{%
    		\usebibmacro{cite:label}%
            \setunit{%
                \global\booltrue{cbx:parens}%
                \printdelim{nonameyeardelim}\bibopenparen%
            }%
            \ifnumequal{%
            	\value{citecount}%
            }{1}{%
            	\usebibmacro{prenote}%
            }{}%
            \usebibmacro{cite:labeldate+extradate}%
        }{%
        	\usebibmacro{cite:shorthand}%
        }}{%
        \iffieldundef{shorthand}{%
        	\printnames{labelname}%
            \setunit{%
                \global\booltrue{cbx:parens}%
                \printdelim{nameyeardelim}\bibopenparen%
            }%
           	\ifnumequal{%
           		\value{citecount}%
           	}{1}{%
           		\usebibmacro{prenote}%
           	}{}%
            \usebibmacro{citeyear}%
        }{%
        	\ifciteseen{%
        		\usebibmacro{cite:shorthand}%
        	}{%
        		\printnames{labelname}%
                \setunit{%
                    \global\booltrue{cbx:parens}%
                    \printdelim{nameyeardelim}\bibopenparen%
                }%
                \ifnumequal{%
                	\value{citecount}%
                }{1}{%
                	\usebibmacro{prenote}%
               	}{}%
                \iffieldundef{labelyear}{%
                	\usebibmacro{cite:label}%
                    \setunit{%
                    	\addcomma%
                    	\addspace%
                    }%
                    \iffieldundef{shorthandintro}{%
	                	\printtext{\bibstring{citedas}}%
	                }{%
	                	\printtext{\printfield{shorthandintro}}%
	                }%
	                \addspace%
                    \usebibmacro{cite:shorthand}%
                }{%
                	\usebibmacro{cite:labeldate+extradate}%
                    \setunit{%
                    	\addcomma%
                    	\addspace%
                    }%
                    \iffieldundef{shorthandintro}{%
	                    \printtext{\bibstring{citedas}}%
	                }{%
	                	\printtext{\printfield{shorthandintro}}%
	                }%
	                \addspace%
                    \usebibmacro{cite:shorthand}}}}}}%
%------------------------------
% ONLY FOR DISSERTATIONS
%------------------------------
\DeclareBibliographyCategory{inbib}	% a specific category for all citations in the text, NOT in the publication list
\newcommand{\printpublication}[1]{\AtNextCite{\defcounter{maxnames}{99}}\fullcite{#1}}						% print all authors in case the \printpublication command is used
\makeatletter										
\AtEveryCitekey{%
  \ifcsstring{blx@delimcontext}{fullcite}	% if a citation is made with the fullcite command, do not show it in the references
    {}
    {\addtocategory{inbib}{\thefield{entrykey}}}}	% if is is normally cited later, add it to the references
\makeatother
%------------------------------
% INTERNET
%------------------------------
\renewcommand*{\UrlFont}{\upshape}
%------------------------------
% HYPERREF
%------------------------------
\hypersetup{%
   	bookmarksnumbered=true,		% bookmarks are numbered in PDF
   	bookmarksopen=false,			% bookmarks are closed when PDF is opened
   	citebordercolor=blue,		% border color for citations
   	colorlinks=false,			% if true: color the font, if false: put boxes around the links
   	hypertexnames=true, 			% use guessable names for links
   	linkbordercolor=red, 		% color for links
   	linktocpage, 				% use page numbers as links in table of contents    
   	pdfpagelayout=TwoPageRight, 	% automatically open PDF as double page, odd pages to the right
   	urlbordercolor=cyan,			% border color for URLs
}%
%------------------------------
% MAKEINDEX
%------------------------------
\makenomenclature				% generate an .nlo file for the List of Symbols
\iflanguage{ngerman}{
	\newcommand{\indextitle}{Indexverzeichnis}
}{
\iflanguage{english}{
	\newcommand{\indextitle}{Index}
}{}
}
\makeindex[
	columns=2,					% number of columns
	intoc=true,					% add to List of Contents
	options= -s 05_Literature_and_Index/myindexstyle.ist,	% index style file
	title={\indextitle\label{ch:index}},	% title of the Index
	]							% generate an .idx file for the Index
\setindexpreamble{\label{ch:index}}
%------------------------------
% OWN COMMANDS
%------------------------------
% You can use this area for additional own commands
% in this way, it is simpler to pull a new version of the template - you can simply copy this area to the new version
% make sure that these commands do not require or override commands from above